\chapter{Methodological approach}
- (3h) Writing Process: Go through markdown notes (prepare writing)
- (2h+) Refactoring: Start Camunda Modeler (see whether it works)
	-> Look at results for next steps


\section{Motivation and Procedure}
% 
% What we did so far: Transition
% 

%
% Goals and Priority
%


% Primary Goal is to do the Refactoring Process

% TODO: Formulate this caveat after writing the business case
	% Business Case will not be covered, technical debt will not be estimated.
	% - But: Software Project will be transformed in future
	% - Thus: By the detection of internal problems coupled future development,
	% 	refactoring seems to be self-evident.

% Note: Reworking the entire Application is not in the scope of the thesis

% Secondary Goals - Discussion part
Whereas primary ...





% ------------------------------------------------------------------------
% Hypotheses
% ------------------------------------------------------------------------
Based on these goals -> constucted following hypotheses.

During the practical work, 
	the thesis will try to validate two types of hypotheses.


% ------------------------------------------------------------------------
% Transition: Short Overview of Procedure
% ------------------------------------------------------------------------

\section{Analysis: Detecting Code Smells}

% Introduction Paragraph
The following part of this thesis moves on to describe in greater detail detection code smells within the software system. It provides a brief overview of the methods used for detection and the mentions the prevelance of sonagraph as the main software utilized in the detection activities. Moreover this section provides some clarifications necessary to comprehend certain methods to be chose over others.

% Catalogue
Out of 24 code smells that are examined throughout Martin Fowler examines book on refactoring, only 10 are chosen to be included in the detection work. Further, the 2 smells \it{refused bequest} and \it{data clumps} were discarded, as their focus on inheritance and data structures respectively, are not utilized in the code. In order to revisit the code smell catalogue with their corresponding explanation, refer to (sec:...)
 

% Restrictions
There exists several reasons that have lead to the decision of restricting the amount of smells for this section. First, it is not the scope of the thesis to find every code smell in the codebase. Finding and evaluating each smell that is known in literature would be a tedious task, as there are no automatic tools available for the python programming language, as will be discussed in a later part of this section. Hence, it suffices, if some smells that are prevelant have been detected. In addition, it is also advantegous to only focus on the most prominent smells. Very niche smells could distract improving the quality of the entire code base, as well as provide less of possibilities to make generalizations.


\section{Testing: Framework}

\section{Measurment: Success}

\section{Implementation: Refactor}

\section{Extra Activities}

\section{Limitations and justification}
