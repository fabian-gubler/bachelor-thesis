\chapter{Introduction}
Kent Beck stated “I am not a great programmer; I am just a good programmer with great habits” (\cite{fowler2012}). One of these great habits is the ability to write code that is not only written for machines, but can easily be read, modified, and extended by humans. Corresponding programming tasks carried out by humans are prohibited by internal problems existing in the code base, commonly referred to as code smells. Here, refactoring is frequently prescribed to solve these issues. 

Various work has been conducted on refactoring and code smells (\cite{fowler2018}; \cite{lacerda2020}). Several studies have been published on the detection of code smells (\cite{chen2016}; \cite{menshawy2021}) and techniques suitable for refactoring code (\cite{fowler2018}; \cite{mens2004}). These studies have consistently found a close relationship between code smells and refactoring. Despite this, very little has been done to explore how refactoring is related to other fields of software engineering and beyond.

The primary aim of this thesis is to investigate, whether further areas of software engineering can be found that closely relate to refactoring. In addition, the thesis does not limit itself to software engineering practices and also sets out to examine refactoring from a business perspective. The investigation follows a case-study approach, utilizing a software system in the domain of industry 4.0. Combining established literature and an in-depth analysis of the software system, the thesis aspires to develop an overarching framework, which will be referred to as the \emph{refactoring process}. The purpose of this approach is to argue against the notion of refactoring being a distinct activity. Hence, this investigation strives to explore and extend the current understanding of refactoring. The subsequent findings aim to provide a basis for other refactoring endeavors and particularly offer advice related to software systems in industry 4.0.

The thesis is composed of six themed chapters. The first chapter provides an overview of the case study. The aim of the chapter is to introduce the context of industry 4.0 and present the learning factory and its corresponding software system. This chapter previews some of the connections that are to be drawn from refactoring. The second chapter gives a brief overview of the theoretical framework of refactoring. The main topics covered in this chapter are the benefits, criteria, and challenges of refactoring. In addition, the most prominent code smells are introduced and the relationship to refactoring is discussed. Further, the following chapter will examine refactoring by introducing the concept of technical debt. Here, a close connection between refactoring and business will be demonstrated. The fourth chapter identifies new areas that are tightly related to refactoring, extending the scope of the term refactoring. Additionally, the software system is taken as a reference to develop a suitable refactoring methodology. Following, chapter five offers concrete results using the developed methodology from the previous chapter. At this point, the prevalence of code smells is examined and corresponding refactoring measures discussed. At last, the thesis closes with a conclusion, which summarizes the main findings and discusses their relevance. 