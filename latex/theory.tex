\chapter{Theoretical Framework for Code Refactoring}

\section{Background}

% ---
% Defining Refactoring
% ---


Refactoring is a well established concept in software development. 
Especially in complex project, 
	cleaning up code is a regular activity and is often referred to as refactoring. 
Despite of its familiarity, 
	the exact definition of the term "refactoring" is not always self evident.
To avoid refering to the term too loosely, 
	it is therefore important to give a precise definition to reference throughout the paper. 

Martin \textcite{fowler2018} managed to formulate a definition that is both short and precise. 
Notably, Fowler is one of the most prominent figure in the field of refactoring,
	which further valides the utilization of his definition.
Fowler defines it the following:

\begin{quote}
\textbf{Refactoring} is the process of changing a software system in a way 
	that does not alter the external behavior of the code yet improves its internal structure.
\end{quote}

One possibility to improve the internal structure is to work on the quality attributes of the program. 
It is important that quality improvements do not change the features, 
	as the exteral behavior of should remain the same. 
By proper refactoring, we can thus eleviate the risk associated with changing the code,
	while improving the maintainablity of the program.

% ---
% Objective: Explain External behavior and Internal Structure
% ---





\section{Formalization of Design Patterns}
\section{The Business Case for Refactoring}
\section{Thesis Context}
