\chapter{Theoretical Framework for Code Refactoring}

\section{Background}

% ---
% Defining Refactoring
% ---


Refactoring is a well established concept in software development. 
Especially in complex project, 
	tasks such as cleaning up code are a regular activity and often referred to as refactoring. 
Despite of its familiarity, 
	the exact definition of the term "refactoring" is not always self evident.
To avoid refering to the term too loosely, 
	it is therefore important to give a precise definition to reference throughout the paper. 

Martin \textcite{fowler2018} managed to formulate a definition that is both short and precise. 
Notably, Fowler is one of the most prominent figure, 
	who has pioneered many concept in the field of refactoring,
	which validates the utilization of his definition.
Fowler defines refactoring the following:
\begin{quote}
\textbf{Refactoring} is the process of changing a software system in a way 
	that does not alter the external behavior of the code yet improves its internal structure.
\end{quote}

It is therefore an integral part of reactoring to not change the features of the program, 
	while improving the quality of the code.
By proper refactoring, we can thus eleviate the risk associated with changing the codebase,
	while improving the internal structure of the program.

% ---
% Objective: Explain External behavior and Internal Structure
% ---

Spending valuable ressources, while not adding new features, 
	might sound unappealing to many managers and even programmers.
\textcite{kim2021} mentions the problem of not having an immediate benefit when refactoring, 
	unlike new features or bug fixes.
In addition, the value of improving the internals of the code, is hard to show to a manager, 
	who is not an expert and even harder to present to the client, who is paying for the work.



\section{Formalization of Design Patterns}
\section{The Business Case for Refactoring}
\section{Thesis Context}
