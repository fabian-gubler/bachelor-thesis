\chapter*{Abstract}
\label{cha:abstract}

Refactoring is known to improve the quality of a software system by restructuring computer code. Here it is well established that refactoring is in close relationship with the detection of code smells.
This thesis aims to extend the current understanding of refactoring by examining what further connections can be drawn in areas of software engineering and business. Specifically, it develops a process that positions refactoring in a larger context by portraying its key drivers.

To test whether refactoring is a joint activity, the investigation follows a case-study approach, utilizing a software system in the domain of industry 4.0. In doing so, the thesis presents a comprehensive refactoring methodology suitable for this particular case and displays the limitations. The findings provide insights to guide practitioners in their refactoring endeavors and offer practical advice for refactoring software systems specific to industry 4.0.

The results show refactoring having multiple connections to other domains and present an interconnected refactoring process. Findings suggest the detection of code smells, testing and quality measurements as substantial contributors to refactoring success. In addition, the thesis suggests businesses to establish a business case before refactoring and presents and recommends communicating their decisions using the concept of technical debt.
