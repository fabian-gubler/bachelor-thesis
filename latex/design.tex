\chapter{Thesis Context and Design}
\section{Fischertechnik Factory Control}

[Would be nice to have a picture of the Factory here]

% Transition
In the section below, a smart factory produced by Fischertechnik and a software system that controls it will be presented to provide context for upcoming sections.
Although not yet introduced, both will be referred to numerous times throughout the following sections. First, the hardware is introduced very briefly. Second, the software system will be covered in more detailed. We will focus more on the software component, as many of the concepts are primarily applicable to the software system. 

% Restriction
The following parts take on a more practical approach than the previous parts. The thesis focuses on one particular software system. By means of this restriction, it should be noted that as a consequence, we reach conclusions that might not be applicable to all software systems. Although certain assumptions will not be generalizable to all software system, through this approach, we are able to provide much more practical insights and are able to present detailed description on the basis of the application. Additionally, this restriction requests to take design decisions on the preconditions the software system demands. Forced to take such decisions further provides valuable insights to a more practical setting.

% Software Project: Purpose
One specific restriction of the software system is its deep integration to hardware components. The factory, being a cyber-physical system (CPS), is bound to its physical environment. The present software system enables users to interact with the factory with HTTP requests, including executing certain events and getting sensory information. These smaller interactions in turn allow the creation of complex business processes. Further, multiple processes have been formalized through Business Model and Notation (BPMN), allowing these processes to be automatically executed. 

[This part must definitely be expanded, in order to provide a better overview to the reader]
[Industry 4.0 was not explained nor mentioned throughout the thesis (even though it is in the title)]

\section{Motivation and Objectives}
% Primary Motivation
As was evident in the theoretical framework and the business case, the sphere of refactoring should not be limited to the scope of the mere implementation of practical refactoring measures. We saw that refactoring influences various aspects of the software development process and offers multiple opportunities from a business perspective. 

In addition to the theoretic contributions made in the previous chapter, the primary aim of the following work is to provide a comprehensive view on refactoring. It will be argued that refactoring must be placed in a much broader context, having multiple software domains, such as the detection of smells, testing, and measurements. The emphasis will be put on connecting several aspects into one model, the importance of each domain in regard to refactoring will be discussed. These insights aim to provide a basis for other refactoring endeavors and should encourage discussions to further explore the topic. Moreover, the thesis should offer applicable advice directly related to similar software systems in the area of CPS. It is not the focus of this thesis to deliver a refactored product, nor to record details of practical execution. On the contrary, the thesis focuses on exploring the intricate essence of refactoring and tries to discover relationships between different aspects within the context of software development.

% Academic Work
In order to provide this comprehensive view on refactoring, the thesis follows several objectives. On a conceptual level, the thesis aspires to develop an illustrative model of a refactoring process, where refactoring itself is only one component of a broader paradigm. Here, it is important to present how other parts of the software development process influence the success of refactoring. By having an understanding of this model should promote an understanding that separate activities are a fundamental prerequisite to appropriately refactor code. In other words, an argument should be provided that refactoring can not be thought as a distinct activity. Along with developing a refactoring process, the thesis has the objective to develop an appropriate method for each stage of this model. Methods are chosen in conjunction with the software system at hand. Here, the design decisions that led to the selection of the methods should be clearly visible. This is vital, to allow transferring the acquired knowledge to other projects. Lastly, an important objective for the thesis is to think critically of each decision by discussing limitations of the methodological approach provided. The processes effectiveness and validity should not only be measured in regard to the present software system, but also measured with respect to the generalizability to software systems in general.
