\chapter{Conclusion}
This thesis aimed to identify additional areas in software engineering that relate to refactoring. Based on a case-study approach, an overarching framework was developed to present a refactoring process. The results reveal that refactoring is not a distinct activity having significant influences from the detection of code smells, testing environment and the measurement of quality improvements after refactoring. This thesis has also shown many connections between refactoring and the business sphere. Technical debt as a concept has presented the financial incentives of refactoring and proved to be a suitable communication device to justify refactoring decision-making. 

The in-depth analysis of selecting appropriate methods for our case study has shown that many refactoring methods are not generalizable to other software systems. It was highlighted that the programming language influences the availability of automatic tools. Moreover, CPS have presented challenges in testing due to their physical environment and limitations to automate testing. The thesis raised an important interest in measuring the improvement of quality attributes after refactoring. It introduced the maintainability index as a possible measurement technique and presented the importance of taking continuous and automated measurements.

In the detection of code smells, the most obvious internal problem in the software system was the duplication of code. Further, metrics indicated the possibility of two additional smells: Long method and Lazy Class. Not enough evidence was found to reliably suggest refactoring them. In addition, the thesis examined the modular structure of the software system. It revealed indications that support the feasibility of a migration towards microservices. 

Finally, an attempt was made to illustrate the refactoring of code duplication. Starting with just one example, the thesis could propose various structural changes to improve the software system. The results of this illustration suggest refactoring to be a dynamic activity.

The present thesis extends our knowledge of the term refactoring by offering a more comprehensive view. This new understanding should help guide practitioners to make practical decision in their refactoring endeavors. More broadly, by having presented key drivers of refactoring, the thesis offers new opportunities in the development of comprehensive refactoring approach for both software engineering and business. 